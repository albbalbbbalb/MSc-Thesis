\subsection{Perturbations of linear motion}

Recall, that a particle in the plane moves along Larmor circles when in the presence of a uniform magnetic field, with strength $b$, orthogonal to the plane of motion. The Larmor radius $R$ depends inversely with respect to the field strength, $Rb\varpropto 1$, that is, a weaker the field strength corresponds to a larger Larmor radius. So, as $b\to 0$, we expect $R\to\infty$, which means that locally the trajectory approaches linear motion in the plane. This intuition is coroborated when considering the Hamiltonian \eqref{eq:magnetichamiltonian} and setting $b=0$. Hence, it is natural to consider the Hamiltonian $H_0(q,p)=\|p\|^2/2$ and perturb it.

We see $H_0$ is real analytic, it is also in action-angle coordinates from which we can deduce it is non-degenerate $\det \partial_p^2 H_0=1\neq0$, and the frequency map $\partial_p H_0(p) = p$ is a diffeomorphism. Now, we attempt to isolate $H_0$ in \eqref{eq:magnetichamiltonian} from the perturbation term. Writing $A = (A_1,A_2,A_3)$, we see:
\begin{align*}
H(q,p) = \frac{1}{2}\left\|p-A(q)\right\|^2
  &= \frac{1}{2}\left(p_1+ A_1(q)\right)^2 + \frac{1}{2}p_2^2\\
  &= \frac{\|p\|^2}{2} + \frac{1}{2}\left(2p_1A_1(q)+A_1^2(q)\right)\\
  &\stackrel{(\star)}{=} H_0(p) + bH_1(q,p,b)
\end{align*}
where in $(\star)$ we used that we can factor out $b$ from $A_1(q)$. As previously discussed, $H_1$ is discontinuous, so to apply KAM, we need to mollify $H_1$. Hence, consider the mollified perturbation $\hat H_1 = \varphi_\varepsilon*H_1$, where $\varepsilon>0$ is a parameter independent of $b$. Now, by \Cref{thm:KAM}, for sufficiently small $b>0$ there are tori of $H_0$ that are preserved under the perturbation $H_1$.

