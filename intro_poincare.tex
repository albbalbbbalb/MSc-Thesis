In this section we consider the dynamics of \eqref{eq:magnetichamiltonian} for $b\gg 0$, that is, in the case where KAM and perturbative methods are not readily applicable. We approach the system in an exploratory way: we look for (quasi)-periodic orbits, consider their stability, and see where stability is missing.

To this end, we prove the existence of a Poincar\'e section, later we induce ``coarse'' symbolic dynamics and apply the Lempel-Ziv complexity to make sense of the dynamics. What we find is rich dynamics and a visual method of analysis well suited for similar problems.

The phase space of our system is $\mathbb R^2\times\mathbb R^2$, however we can reduce the phase space to a Poincar\'e section as follows:
\begin{proposition}\label{prop:poincaresurface}
Let $S$ be the union of discs of radius $R$ centered at $\mathbb Z^2+1/2$. The sets $\Sin$ and $\Sout$ defined as:
\begin{align}
\label{def:poincaresectionIN}
\Sin &= \bigcup_{x\in\partial S} \{x\} \times \{v\in\mathbb R^2 : v\cdot (x-1/2) < 0\},\\
\label{def:poincaresectionOUT}
\Sout &= \bigcup_{x\in\partial S} \{x\} \times \{v\in\mathbb R^2 : v\cdot (x-1/2) > 0\},
\end{align}
are Poincar\'e sections for the system \eqref{eq:magnetichamiltonian}.
\end{proposition}
We prove this later. This helps visualize long term behavior, since the magnitude of the velocity of a trajectory is constant. Furthermore, we can pass the system to the torus, which reduces $S$ to one disc. So, when plotting we only need two dimensions: one to parametrize $\partial S$ and another for the velocity.