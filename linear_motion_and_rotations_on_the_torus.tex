\subsection{Linear motion and rotations on the torus}

Here we discuss solutions of the Hamiltonian $H:\mathbb R^2\to\mathbb R^2$ with $H(q,p)= \|p\|^2/2$ in the plane and in the torus. The equations of motion can be computed readily from the Hamiltonian system, they are:
\begin{align}\label{eq:linearmotion}
V_1(t) = (q_1(t),q_2(t)),\qquad
q_1(t) = p_1t+q_1(0),\qquad
q_2(t) = p_2t+q_2(0)
\end{align}
where $p_1,p_2\in\mathbb R$ are constants. We understand this motion well, it is just straight lines in the plane.

The situation becomes more interesting when we consider the induced motion on the quotient space $\mathbb T^2=\mathbb R^2\backslash \mathbb Z^2$ via the quotient $P:\mathbb R^2\to\mathbb T^2$ with $P(q_1,q_2) = (q_1\!\!\mod 1,q_2\!\!\mod 1) = [q_1,q_2]$, where the bracket indicates the equivalence class of the point $(q_1,q_2)$. The equations become:
\begin{align}\label{eq:rotationsontorus}
\hat V_1(t) = P\circ V_1(t) = [q_1(t),q_2(t)] = t[p_1,p_2] + [q_1(0),q_2(0)],
\end{align}
so the trajectory that was a line in the plane now wraps around the torus. It is now important to understand how the ratio of $p_1$ to $p_2$ influences the dynamics. 

\begin{definition}
Let $p_1,p_2\in\mathbb R$, we say $p_1,p_2$ are \textit{rationally dependent} if there exist $n,m\in\mathbb Z$ such that $p_1n+p_2m=0$.
\end{definition}

Two numbers are then called \textit{rationally independent} if there does not exist a pair of integers satisfying the equation. This definition helps us formulate the following well-known result.

\begin{proposition}\label{prop:rotationsontorus}
Consider the trajectory $\hat V_1$ of \eqref{eq:rotationsontorus} in the torus.
\begin{enumerate}
\item If $p_1$ and $p_2$ are rationally dependent, then $\hat V_1$ is a periodic solution, and its orbit $\hat V_1([0,\infty))$ is a closed curve on the torus,
\item If $p_1$ and $p_2$ are rationally independent, then $\hat V_1([0,\infty))$ is dense in the torus.
\end{enumerate}
\end{proposition}

Another result that can be mentioned here but will not be important in our discussion is that due to \Cref{prop:rotationsontorus} the set of periodic orbits is dense in phase space.

