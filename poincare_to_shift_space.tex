To study a discrete dynamical system in terms of symbolic dynamics one usually deduces a Markov partition of the space such that the system in question can be related via a (semi)-conjugacy to a shift-space of finite type. What is obtained is a simpler representation encoding the original dynamics. In the previous section, we successfully reduced the dynamics of \eqref{eq:magnetichamiltonian} to a discrete system on either $\Sin$ or $\Sout$. From now on we focus only on $\Sout$.

A first attempt at establishing symbolic dynamics in our case runs into a few issues. Firstly, the shift map is continous on a shift-space, so provided $\Sout$ has the natural product topology, the map $P_\text{o}:\Sout\to\Sout$ should be continuous as well. Recall, $P_\text{o} = P_\text{io}\circ P_\text{oi}$, the map $P_\text{io}$ can be shown to be continuous, however $P_\text{oi}$ is definitely not continuous. It is not clear a priori whether $P_\text{o}$ is continuous, there could exist values of $b$ which make it so. However, it is reasonable to assume that in general $P_\text{o}$ is discontinuous. So, we should find a different topology on $\Sout$ for this method to work.

Next, it seems constructing a Markov partition is difficult. In the construction, we need to consider the stable and unstable sets of points in $\Sout$ under iteration of $P_\text{o}$, and by definition $P_\text{o}$ must be at least differentiable, which we know it is not. Ignoring this issue, we can look for any reasonable partition that does not have the Markov property. 

A reasonable partition should be finite, this way we can salvage some intuition from the above method. Focusing on the effects of $P_\text{oi}$, we see that, with the right conditions, a trajectory could jump to one of an infinite number of other circles relative to the circle we start from. So, to have a finite partition, some concessions must be made. We could partition in such a way that jumps to a circle past a certain radius are all assigned the same symbol. One could also partition the circle lattice in a checkerboard fashion, yet neither of these approaches seem ``natural''. Hence, given all the resistence, we should consider another approach to symbolic dynamics.

Let $a_0=(\theta_0,\varphi_0)\in\Sout$, and define $a_n=P_\text{o}^n(a_0)$, that is, $a_n$ is the $n$-th iterate of $a_0$. In the plane, each circle of $\Sout$ is associated with a coordinate $\mathbb Z^2+1/2$, then to each $a_n$ we associate the coordinate $s_n$ of the circle which $a_n$ is on. Now, to the sequence of iterates $\{a_n\}_{n\ge 0}$ we associate the sequence of symbols $\{s_{n}-s_{n-1}\}_{n\ge1}\subseteq \mathbb Z^2$. In other words, the symbols encode the relative jumps of the trajectory. Immediately, we notice that the alphabet we picked is countably infinite which we previously asserted was not \textit{reasonable}, however we argue that this choice makes the least assumptions and hence in a sense is natural. Our choice will be further justified once we introduce the Lempel-Ziv complexity of strings.