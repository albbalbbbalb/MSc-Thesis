\subsection{Uniform magnetic vector field in the plane}

We focus now on \eqref{eq:magnetichamiltonian} and take $A(q)=(-\varepsilon q_2,0,0)$ and $\varepsilon>0$ as used for \eqref{eq:magnetichamiltonian2}, except, we do not include the indicator $\one_S$. This gives the Hamiltonian for the motion of a particle in the $(q_1,q_2)$-plane in the presence of a vector field in the plane.
\begin{align}
H(q,p) =
\frac{1}{2}(p_1+\varepsilon q_2)^2+\frac{1}{2}p_2^2,
\end{align}
From $H$, we find Hamilton's equations:
\begin{align*}
\frac{dq_1}{dt} &= 2(p_1+\varepsilon q_2) 	& \frac{dq_2}{dt} &= 2p_2\\
\frac{dp_1}{dt} &=0							& \frac{dp_2}{dt} &=-2\varepsilon(p_1+\varepsilon q_2).
\end{align*}
And we can find the equations of motion: $X_2(t) = (q_1(t),q_2(t))$ with
\begin{align*}
q_1(t) = \alpha_1 - R\cos(\varepsilon t+\varphi),\qquad 
q_2(t) = \alpha_2 + R\sin(\varepsilon t+\varphi)
\end{align*}
where $\alpha_1,\alpha_2,R,\varphi$ are integration constants. These can be computed explicitly:
\begin{align*}
\alpha_1 &= q_1(0)+\frac{p_2(0)}{\varepsilon} &
		R &= \frac{1}{\varepsilon}\|X_2'(t)\|\\
\alpha_2  &= q_2(0)-\frac{p_1(0)}{\varepsilon} &
		\varphi &= \text{arctan2}\,(p_1(0),p_2(0))
\end{align*}
where $\text{arctan2}$ behaves like $\arctan$ with the added benefit that it correctly determines the quadrant of the angle. From the equations of motion, we see that the orbits are circles which we call \textit{Larmor} circles. The \textit{Larmor} radii $R$ and centers  $(\alpha_1,\alpha_2)$ of the circles are determined uniquely by the initial conditions of the particle as we can see above.
